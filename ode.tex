\documentclass[dvipdfmx,autodetect-engine]{jbook}
\usepackage{tikz}
\usepackage{amsmath}
\usepackage{amsthm}
\usepackage{amssymb}
\usepackage{amsfonts}
\usepackage{nccmath}
\usepackage{tikz-cd}
\usepackage{tcolorbox}
\usepackage{graphicx}
\usepackage{wrapfig}
\usepackage{here}
\usepackage{fancyhdr}
\definecolor{astral}{RGB}{46,116,181}

\newcommand{\R}{\mathbb{R}}
\newcommand{\disk}{\mathbb{D}}
\newcommand{\Imdisk}{f(\mathbb{D})}
\newcommand{\relmiddle}[1]{\mathrel{}\middle#1\mathrel{}}

\tcbuselibrary{theorems}
\newtcbtheorem[number within=section]{mytheo}{}{coltitle=astral,width=\linewidth,arc=0mm,colback=astral!20!white,colframe=astral!20!white,colframe=astral!20!white,fonttitle=\bfseries}{th}

\newtcbtheorem[number within=section]{mydef}{}{coltitle=orange,width=\linewidth,arc=0mm,colback=orange!20!white,colframe=orange!20!white,colframe=orange!20!white,fonttitle=\bfseries}{th}

\begin{document}
\pagestyle{fancy}
  \fancyhead{}
  \fancyhead[RO,RE]{\rightmark}
  \fancyhead[LE,LO]{\leftmark}
  \cfoot{\thepage}
  \renewcommand{\chaptermark}[1]{\markboth{第\ \thechapter\ 章~#1}{}}
  \renewcommand{\sectionmark}[1]{\markright{\thesection #1}{}}

 \chapter{ ボホナー積分}
 ルベーグ積分の理論はリーマン積分を自然に拡張し,収束定理をはじめとする多くの実り豊かな結果をもたらしてきた.
 ルベーグ積分は実(複素)数値関数に対して定義され,さらに(有限次元)ベクトル値関数へと自然に拡張される.
 そのさらなる一般化としてバナッハ空間値関数に対する積分が定義できるかということが問題となる.
 その一つの定式化を与えるのがボホナー積分である.\par
 この章ではボホナー積分を簡潔に導入し,微積分学の基本定理を示すことを目標とする.

\section{ボホナー積分の定義}
  \begin{mydef}{\textbf{Notation}}{not1}
    \begin{itemize}
      \item $(E,\|\ \|)$はバナッハ空間.
      \item $\mathcal{B}(X)$は$X$のボレル集合族.
      \item $(\Omega , \mathcal{F},\mu)$は$\sigma$-有限な測度空間.
    \end{itemize}
  \end{mydef}
 まずは単関数の積分を定義する.
  \begin{mydef}{\textbf{Definition}}{def1}
    単関数からなる集合
    \begin{equation*}
      \mathcal{E}:=\left\{ f:\Omega \to X \relmiddle| f=\sum_{k=1}^{n}x_{k}\chi_{A_k},x_k \in E , A_k \in \mathcal{F} , \mu(A_k)< \infty  \right\}.
    \end{equation*}
    ベクトル空間$\mathcal{E}$のセミノルムを次で定める.
    \begin{equation*}
      \|f\|_{\mathcal{E}}:=\int \|f\|d\mu.
    \end{equation*}
    $(\mathcal{E},\|\ \|_{\mathcal{E}})$を$\|\ \|_{\mathcal{E}}$についての同値関係で割ることでノルム空間をえるが,簡潔のため同じ記号$(\mathcal{E},\|\ \|_{\mathcal{E}})$で表す.
  \end{mydef}
  \begin{mytheo}{\textbf{Proposition}}{theo1}
    任意の$f\in \mathcal{E}$は
    \begin{equation*}
      f=\sum_{k=1}^{n}x_k \chi_{A_k},\ \ \ A_{k}は互いに素
    \end{equation*}
    と書くことができる.
  \end{mytheo}
  \begin{proof}
    $f$は単関数であり,$f(\Omega)=\{x_1,\ldots,x_n\}$($x_i$は相異なる)について$A_k:=f^{-1}(x_k)$とおけば求める表示を得る.
  \end{proof}

  \begin{mydef}{\textbf{Definition}}{def2}
    $f=\sum_{k=1}^{n}x_k \chi_{A_k}$($A_k$は互いに素)に対し,
    \begin{equation*}
      \int f d\mu := \sum_{k=1}^{n}x_k \mu\left(A_k\right).
    \end{equation*}
  \end{mydef}

  この定義が$f$の表示及び$\|\ \|_{\mathcal{E}}$についての同値関係によらないことを確認する.

  \begin{mytheo}{\textbf{Proposition}}{theo2}
    $\int f$,$f\in \mathcal{E}$は$f=\sum_{k=1}^{n}x_k \chi_{A_k}$($A_{k}$は互いに素)の表示によらない. \par
    また$\|f\|_{\mathcal{E}}=0$なら$\int f=0$である.\par
    したがって線形写像
    \begin{equation*}
      \begin{array}{ccc}
          \rm{int}:\ \ (\mathcal{E},\|\ \|_{\mathcal{E}}) & {\longrightarrow} & (E,\|\ \|) \\
           \ \ \ \ \ \ \ \ f & \longmapsto & \int fd\mu
      \end{array}
    \end{equation*}
    が定まる.
  \end{mytheo}
  \begin{proof}
    $f(\Omega)=\{y_1,\ldots,y_m\}$($x_i$は相異なる)について$B_k:=f^{-1}(y_k)$とおく.$A_{k}$は互いに素であり$f$は単関数なので
    \begin{align*}
      \sum_{k=1}^{n}x_k \mu\left(A_k\right) &= \sum_{k=1}^{n} \sum_{l=1}^{m} x_k \mu\left(A_k\cap B_l\right) \\
                                            &= \sum_{l=1}^{m} \sum_{k=1}^{n} y_l \mu\left(A_k\cap B_l\right) \\
                                            &= \sum_{l=1}^{m} y_l \mu\left(B_l\right)
    \end{align*}
    よって$f$の表示によらない.\par
    $\|f\|_{\mathcal{E}}=0$なる$f\in\mathcal{E}$について,$f=\sum_{k=1}^{n}x_k \chi_{A_k}$($A_{k}$は互いに素)と書いた時
    \begin{align*}
      \int \|f\| d\mu = \sum_{k=1}^{n}\|x_k\|\mu\left(A_k\right) = 0
    \end{align*}
    なので$\int f d\mu = \sum_{k=1}^{n}x_k\mu\left(A_k\right)=0$を得る.
  \end{proof}

  \begin{mytheo}{\textbf{Proposition}}{theo3}
    $f\in \mathcal{E}$に対し$\|\int fd\mu\|\leqq\int\|f\|d\mu$. \par
    よって$\rm{int}:\mathcal{E}\to E$は有界線型写像であり,拡張原理より$\rm{\overline{int}}:\mathcal{\overline{E}}\to E$を得る.
  \end{mytheo}
  \begin{proof}
    三角不等式より
    \begin{align*}
      \| \int f d\mu \| \leqq \sum_{k=1}^{n}\|x_k\|\mu\left(A_k\right) = \int \|f\| d\mu.
    \end{align*}
  \end{proof}
  次に$\mathcal{\overline{E}}$の具体的な表示を求める.

  \begin{mydef}{\textbf{Definition}}{def3}
    関数$f:\Omega \to E$が強可測関数であるとは,$\mathcal{F}/{\mathcal{B}\left(E \right)}$-可測かつ$f(\Omega)\subset E$が可分であることをいう.
  \end{mydef}

  \begin{mydef}{\textbf{Definition}}{def4}
    $1\leqq p < \infty$に対し,
    \begin{align*}
      \mathcal{L}^p \left( \Omega,\mathcal{F},\mu ; E \right) = \left\{ f:\Omega \to E \relmiddle| fは強可測関数 , \int \|f\|^p d\mu < \infty \right\}
    \end{align*}
    そしてセミノルムを
    \begin{align*}
      \|f\|_{L^p} := \left( \int \| f \|^p d\mu \right)^{\frac{1}{p}}
    \end{align*}
    と定める.$\mathcal{L}^p \left( \Omega,\mathcal{F},\mu ; E \right)$を$\|f\|_{L^p}$についての同値関係で割って得られるノルム空間を
    $\left(L^p \left( \Omega,\mathcal{F},\mu ; E \right), \|f\|_{L^p} \right)$と書く.
  \end{mydef}



%  $L^1 \left( \Omega,\mathcal{F},\mu ; E \right):= \mathcal{\overline{E}}$を示すために,いくつか準備をする.

%  \begin{mytheo}{This is my title}{theoexample}
%        aa
%  \end{mytheo}
%  \begin{mydef}{\textbf{Definition}}{def1}
%    aa
%  \end{mydef}

\end{document}
